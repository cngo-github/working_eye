\Section{Background}
\label{sec:background}

The introduction of the internet to the masses led to the birth of many new
businesses and business models for traditional businesses. The growth of
e-commerce, or the purchasing and delivery of goods and services over the
internet, has been astounding. From 2001 to 2002, e-commerce revenues jump by
$37\%$, hitting $\$70$ billion USD \cite{cheng2006adoption} and today's biggest
companies are in the e-commerce space, including Google and Amazon.

Banks, traditionally a business that has relied on having a physical presence 
in the areas that it serves, have also made the jump into the e-commerce 
space with some banks, like USAA, existing only or mostly on the internet. 
Online banking offers a number of advantages over the traditional model of
banking, including greater speed, more flexibility in when the services can be
carried out, greater accessibility in remote regions of the world, and lower
operational costs yielding additional savings for the business
\cite{oghenerukeybe2009customers}. This has helped spurred along the adoption of
online banks since the early 2000s. For the remainder of this paper, online
banks will refer to all banks that conduct their business, in part or in whole,
over the internet.

In order to enable its customers to use the services provided by the bank over
the internet, online banks require the user to authenticate themselves prior to
use. This is done using a primary authentication system, like a user name and
password, with a backup fallback system, like password retrieval challenge
questions \cite{kleuckerfallback}. This, of course, makes online banks
susceptible to a number of security threats that affect e-commerce businesses in
general \cite{kleuckerfallback} and the businesses need to provide an assurance
of the security of private information.

While this is important for all 
businesses that use information systems \cite{suh2003effect} 
\cite{cheng2006adoption} \cite{oghenerukeybe2009customers}
\cite{aladwani2001online}, it is particularly relevant to online banks
\cite{oghenerukeybe2009customers}. This, of course, means that the
authentication systems used by online banks must be secure, or at least 
perceived to be secure. Much emphasis has been placed on the security of the
primary authentication system, but the fallback authentication system also
needs to be secure. Ben Smyth \cite{smyth2010forgotten} examined the
various authentication systems used by the Lloyds Banking Group and the Royal
Bank of Scotland and found them to be lacking.

In this paper, the author
intends to examine the state of modern security used by online banks to see
what has changed since the study by Smyth. In order to evaluate security,
the authors followed the procedure set forth by Just et al.
\cite{just2009personal} in their 2009 paper.